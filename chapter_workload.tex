\chapter{Workload Metrics} \label{ch:workload}

The paper presented in chapter 2 was written before the workload concepts and modeling language semantics had settled into their current form.  Additionally, Jared et al., in concert with this work, published a paper in the FVHMS proceedings that expands on the concepts in chapter 2 but is also slightly outdated.  Thus we feel it expedient to summarize those aspects of the language which are critical to our metrics before we present the metrics themselves.  To assist in this effort we have prepared a simple scenario which includes a partial model and illustrations of the DiRG, DiTG, and labeled state transition system.

\section{Summary of workload concepts}
Show the labeled state transition system and talk about how it represents the scenario.

\subsection{Example Scenario}

In this scenario there are two main actors, Alice and Bob.  Alice is standing next to Bob listening to a friend on her cell phone.  Bob suddenly remembers that he wants to ask Alice out.  Not noticing that Alice is listening to her phone Bob starts to ask Alice on a date.  After listening to both her phone and Bob for a minute Alice looks at Bob and points to her phone.  Bob stops talking and decides between waiting for her to finish or walking away.  Eventually Bob walks away.

From the scenario above we chose to create two Actors, Alice and Bob.  Actors represent any aspect of the system that has state.  While in this scenario both our Actors are human, an Actor can be anything.  For example we could add lighting to the scenario and give it a light and a dark state.  We could also create a sub-Actor which is part of a larger Actor, such as Bob's hair, and give it states like messy or combed.  Actors can also be very abstract or very detailed.  The more states an Actor contains, the more expressive it becomes.

\begin{figure}[h]
\begin{center}
\includegraphics[width=\textwidth]{ab_dirg.png}
\caption{Directed Role Graph for Alice and Bob scenario}
\label{fig:ab_dirg}
\end{center}
\end{figure}

We express an Actor as a Directed Role Graph (DiRG).  A DiRG represents how an Actor is allowed to flow between states.  Figure~\ref{fig:ab_dirg} shows the DiRGs for both Alice and Bob as finite state machines.  We see that Alice is initially in a {\em Listening on phone} state while Bob is in the {\em Standing idle} state.  Alice can either stay in the {\em Listening on phone} state, shown by the looping transition, or move into the {\em Signaling Bob} state.  Once in the {\em Signaling Bob} state her only choice is to stay there forever or to return to the {\em Listening on phone} state.  Individually these DiRGs are of little value, together they begin to express the larger system.  We see from the labels on the DiRG that Alice and Bob are interacting with one another and influencing the transitions of the other.  Before we discuss Actor transitions we must first define the inter-Actor relationship that allows Actors to influence one another.

We define these inter-Actor connections as Channels.  A channel is a uni-directional communication medium which allows an Actor to send information to another Actor.  Each Channel is composed of a source Actor, a target Actor, and a type.  The source Actor sends information as {\em output}, the target Actor receives the information as {\em input}, and the type specifies which communication medium is being used.  In the case of Alice and Bob we use audio and visual Channels\cite{wickens2002multiple}, the case study in chapter \ref{ch:UASinNAS} also uses a Data channel that represents network communication.  We also designate that each Channel can represent multiple layers of communication.  To show this we will use the visual channel from Alice to Bob as an example.  We can express Alices {\em output}, Bobs {\em input}, as two different layers on the visual channel, one for Alices body language, and another for her facial expressions.  This allows us to explicitly set how much data is being sent over the channel, it also allows us to express multiple visual inputs for Bob without creating a {\em channel conflict}.  A {\em channel conflict} occurs when an Actor is receiving input from two or more channels of the same type.  In our example scenario Alice is listening to her phone which uses an audio channel.  At the same time Bob is talking to Alice on a different audio channel.  Because Alice is receiving {\em input} on multiple audio channels she has a channel conflict on her audio channel.  

\subsection{DiTG}
To express a systems Channels we use a Directed Team Graph (DiTG)\cite{FVHMS}.  The DiTG defines all the channels that exist between the Actors within the system.  Figure \ref{fig:ab_ditg} shows the DiTG for our Alice and Bob scenario.  From the figure we can see that Alice has two channels to Bob, an audio and a visual.  We also see that Bob has two channels to Alice, an audio and a visual.  While the simplicity of this scenario makes it difficult to see the value of the DiTG, by combining the DiTG with the DiRG we have effectively constrained the Actor behavior and communication for the entire system.  With the constraints in place we are ready to define the behavior of the system, which we do with Transitions.

\begin{figure}[h]
\begin{center}
\includegraphics[width=\textwidth]{ab_ditg.png}
\caption{Directed Team Graph for Alice and Bob scenario}
\label{fig:ab_ditg}
\end{center}
\end{figure}

\subsection{Transitions}
Transitions represent an Actors behavior.  Transitions tells us about an Actors state, what caused the Actor to change state, and how that change effects the system.  Transitions are composed of a start state, an end state, a set of input equations, a set of outputs, a duration, and a priority.  The Transition start and end states are the states of the Actor and must not violate the DiRG.  The Transition input equations are used to determine if the transition is enabled.  Each equation is composed of a source value, a predicate, and an expected value.  The source value is obtained from one of two sources, Channels or Memory.  Actor memory is an internal variable that allows an Actor to store and retrieve data.  For predicates we use equal to, less than, greater than, etc.  The structure of the input equations allows each equation to evaluate to a simple true or false.  If all input equations evaluate to true then the Transition is enabled.  The Transition outputs contain all output generated by the transition as a set of target value pairs.  The target is the Channel or Memory variable that will receive the designated value.  The Transition duration is a range which represents the minimum and maximum number of time steps that the transition will remain {\em active} before it {\em fires}.  When an Actor decides to transition it selects an enabled transition to become {\em active}.  While a Transition is {\em active} the transition outputs are sent out but the Actor does not change state until the Transition {\em fires}.




In the case of Alice and Bob, Alice transitions the {\em Listening on phone} state to a {\em Signaling Bob} state.  

\subsection{Input and Output}
Within the context of our simulations we sometimes refer to channels and inputs/outputs synonymously.  


The Directed Team Graph (DiTG) 


In this and many other cases a single Actor is of little interest.  What is of more interest is a group of Actors working together.  This is expressed as a Directed Role Graph

For this scenario the first step to building our model is to define our Directed Role Graph (DiRG) and our Directed Team Graph (DiTG).  The DiRG is a collection of finite state machines.



\subsection{Labeled State Transition System}


\subsection{Directed Role Graph}
What is a Directed Role Graph.  Show the DiRG for Alice and Bob.
\subsubsection{Actors}
What is an Actor?  What does it represent?  Actor State?  Set of transitions for each state.  

\subsubsection{Events}
What are events? how are they used? intuition?

\subsubsection{Actor Load}
What is it? Intuition behind it?  How is it represented? Memory

\subsubsection{Transitions}
What is a transition?  What is an Active transition? Durations.  Input and Output.

\subsection{Directed Team Graph}
What is a DiTG.  What are the rules behind it.  Show the DiTG for Alice and Bob.

\subsubsection{Channels}
What is a channel?  Properties of a channel?  Intuition of a channel?  Layers of a channel?  Active vs Inactive.  Input vs Output

\subsubsection{Simulator}
Delta clock?  Get next transition.  Then fire the transition.

\section{Resource Workload}
Describe what it is, point reader towards FVHMS paper.  Present the metric itself with the equation.
Intuition behind it?

\section{Decision Workload}
Describe what decision workload is, point reader towards FVHMS paper.  Present the metric itself with the equation.
Intuition behind it?

\section{Adapted Wickens' Metric}
Reason for creating it, intuition.  What it is (copy paste).