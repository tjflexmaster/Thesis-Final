\chapter{Summary and Future Work}

As part of this thesis work we have developed the Model Abstraction Framework.  This modeling framework is based on the concept of directed role graphs and directed team graphs and is capable of modeling complex human machine systems as labeled state transition systems.  We have shown that the Model Abstraction Framework is extremely flexible, allowing each portion of a system to be modeled at varying levels of abstraction.  This allows us to model system components without knowing the implementation details.  We demonstrated this in the UAV Enabled Wilderness Search and Rescue (WiSAR) case study when we found flaws in the Video Operator role that were previously undocumented and again in the Unmanned Aerial System integration into the National Airspace System case study when we demonstrated the effect of automating a portion of the UAV Operator duties.  

Additionally we have demonstrated our ability to perform model checking directly on the Model Abstraction Framework models using Java Pathfinder (JPF).  By creating models directly in Java there is no need for an extra conversion to a model checking platform.  We also have the ability to create multiple paths through the model with the use of transitions, transition duration ranges, and events. While we leave it to future work to fully realize the potential of model checking the Model Abstraction Framework with JPF we believe this forms the basis of an extremely powerful modeling tool.

We demonstrated the ability to implement an XML model parser to facilitate more efficient modeling.  While the improvement the XML provided was not as great as we had hoped this does demonstrate the ability to adapt to more efficient methods of modeling.  One area of future work is to implement conversion classes between our modeling language and some of the other more popular modeling languages such as Brahms or EOFM.  This would give us additional validation options ~\cite{bolton2013litreview}.

Finally we have demonstrated the ability to extract workload metrics that are consistent with known high workload events and closely related to human workload theories such as Multiple Resource Theory, Threaded Cognition Theory, and Operator fan-out~\cite{wickens2002multiple, salvucci2008threaded, cummings2007predicting}.  We are encouraged by the predicted workload reduction seen in our second case study when we used variations in our model to represent the effects of automating a human performed task.  And we will use this idea of predicting workload to reduce actual operator workload as a seque into the future of this work.

\section{Future Work}

The ultimate goal of this work is to create a UAS for WiSAR that allows a single human to perform all of the necessary tasks.  Before this can be accomplished using the Model Abstraction Framework the workload metrics need to be verified through sensitivity and user studies, currently being pursued by other researchers.  We anticipate that machine learning algorithms will be able to take data from these to produce a set of weightings that can be applied to the individual workload metrics.  Once the workload metrics have settled the next step is to polish up the WiSAR model, add performance measurement Actors, and systematically modify the model to find a single human design that satisfies the performance measurements and maintains workload below a certain threshold.  The next step after completing the WiSAR model is to create a generic UAS model that can then be used to explore the effects of novel UAS GUI designs.

In relation to the future of the Model Abstraction Framework there are several avenues of future work that we feel would be the most beneficial.  The first is improvements to the modeling process.  We see this happening in several ways: new language structures that improve the ability to visualize the model, conversion classes that take advantage of existing language structures to generate models for our framework, or extending another modeling language such as Brahms or EOFM to work in a similar fashion.

Another avenue of future work is to establish a verification framework to facilitate the model checking process.  One way we see this working is to allow the models to specify metric thresholds, Actor-specific performance measurements, and dynamic portions of the model.  The verification framework then uses machine learning to iteratively change the dynamic portions of the model to find an optimized design.



