\chapter{Conclusions and Future Work}

As part of this thesis work we have developed a new modeling framework based on the concept of DiRGs and DiTGs which is capable of modeling complex human machine systems as state machines.  We have shown that our modeling framework is extremely flexible allowing each portion of a system to be modeled at varying levels of abstraction.  This allows us to model system components without knowing the implementation details which we demonstrated in the WiSAR case study when we found flaws in the Video Operator role which were previously undocumented and again in the UAS integration into the NAS when we demonstrated the effect of automating a portion of the UAS.  

Additionally we have demonstrated our ability to explore all possible paths through a model by using JPF.  With this tool we were able to find paths in WiSAR which lead to critical failure.  By taking the ability to explore all paths through the model and coupling it with the flexibility of using Java as the base modeling language we have the basis of an extremely powerful modeling tool.

In our latest code re-factor we demonstrated the ability to implement an XML model parser to facilitate more efficient modeling.  While the improvement the XML provided was not as great as we had hoped this does demonstrate the ability to adapt to more efficient methods of modeling.  One area of future work is to implement conversion classes between our modeling language and some of the other more popular modeling languages such as Brahms or EOFM which would give us additional validation options ~\cite{bolton2013litreview}.

Finally we have demonstrated our ability to extract workload metrics which are consistent with known high workload events and closely related to human workload theories such as Wickens multiple resource theory~\cite{wickens2002multiple}.  We are encouraged by the predicted workload reduction seen in our second case study which segues into the future of this work.

From the beginning the goal of this work was to create a UAS for WiSAR which allowed a single human to perform all of the necessary tasks.  Before this can be accomplished using this modeling framework the workload metrics need to be verified through sensitivity and user studies.  We anticipate that machine learning algorithms will be able to take data from the sensitivity and user studies to produce a set of weightings which can be applied to the individual workload metrics.  Once the workload metrics have settled the next step is to polish up the WiSAR model, add performance measurement Actors, and systematically modify the model to find a single human design which satisfies the performance measurements and maintains workload below a certain threshold.  The next step after completing the WiSAR model is to create a generic UAS model which can then be used to explore the effects of novel UAS GUI designs.

In relation to the future of this modeling framework there are several avenues of future work which we feel would be the most beneficial.  The first is improvements to the modeling process.  We see this happening in several ways; new language structures which improve the ability to visualize the model, conversion classes which take advantage of existing language structures to generate models for our framework, or extending another modeling language such as Brahms or EOFM to work in a similar fashion.

Another avenue of future work is to establish a verification framework to facilitate the model checking process.  One way we see this working is to allow the models to specify metric thresholds, Actor specific performance measurements, and dynamic portions of the model.  The verification framework then uses machine learning to iteratively change the dynamic portions of the model to find an optimized design.



