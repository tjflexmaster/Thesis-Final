\chapter{Summary and Future Work} \label{ch:summary}

As part of this thesis work we have developed the Model Abstraction Framework.  This simple modeling framework allowed us to experiment with generating different types of raw data from a human-machine model.  The modeling framework uses the concepts of Directed Role Graphs and Directed Team Graphs to build models constrained by specific behavior and communication.  The framework then takes this model and converts it into a labeled state transition system that is run on the simulator.  During the simulation we collect raw data from the states, transitions, and labels.

Using this raw data we have demonstrated the generation of three workload metrics: Adapted Wickens' Model, Resource Workload Metric, and Decision Workload Metric.  The Adapted Wicken's Model, our baseline metric, is an adaptation of a simple model meant to measure cognitive resource load~\cite{wickens2002multiple}.  The Resource Workload Metric, based off of the cognitive workload category presented by Jared et al. \cite{moore2014modeling}, represents the cognitive resource load from inter-actor communication and memory access.  The Decision Workload Metric, based off of the algorithmic workload category presented by Jared et al. \cite{moore2014modeling}, represents the complexity of the decision making process.

Using these metrics we demonstrated two important results.  The first result is a general consistency between our baseline model, Adapted Wickens' Model, and the combination of the Resource and Decision workload metrics.  While the Resource and Decision workload metrics are more expressive they trend very closely with the Adapted Wickens' Model.

The second result was a general consistency with the known high and low workload periods observed during the UAV Enabled Wilderness Search and Rescue (WiSAR) flight tests.  Indeed the results from the Resource and Decision workload metrics was more closely aligned to our observations than that of the Adapted Wickens' Model.

Additionally we are encouraged by the predicted workload reduction observed when we introduced automation into scenario three of the case study.  The ability to predict workload reduction with models is one of the primary objectives towards reducing the number of human required to operate a UAS.

\section{Threats to Validity}

While these results are encouraging we realize that they do not yet represent human workload.  The Model Abstraction Framework was developed to enable experimentation with the generation of raw data.  No studies have been performed showing that the Model Abstraction Framework can accurately model human-machine systems.  No work has been done to show that the raw data generated by the framework correlates to memory usage, decision making, cognitive resource usage, and other recognized aspects of human workload.

The baseline workload results were generated by the Model Abstraction Framework and not by a separate modeling framework that has been shown to correctly model human-machine systems and cognitive resource load.  Since there is no guarantee that converting the model used by the Model Abstraction Framework to another modeling language produces the same model we chose to instead attempt to convert the workload metric from a task based paradigm to a state/transition based paradigm.

The WiSAR high and low workload profile used for validating workload consistency is not the result of actively measuring different aspects of human workload during a flight test.  Instead the workload profile was obtained from in-flight observations and post-flight surveys for a limited number of flight tests.

\begin{comment}
We have shown that the Model Abstraction Framework is extremely flexible, allowing each portion of a system to be modeled at varying levels of abstraction.  This allows us to model system components without knowing the implementation details.  We demonstrated this in the UAV Enabled Wilderness Search and Rescue (WiSAR) case study when we found flaws in the Video Operator role that were previously undocumented and again in the Unmanned Aerial System integration into the National Airspace System case study when we demonstrated the effect of automating a portion of the UAV Operator duties.  

Additionally we have demonstrated our ability to perform model checking directly on the Model Abstraction Framework models using Java Pathfinder (JPF).  By creating models directly in Java there is no need for an extra conversion to a model checking platform.  We also have the ability to create multiple paths through the model with the use of transitions, transition duration ranges, and events. While we leave it to future work to fully realize the potential of model checking the Model Abstraction Framework with JPF we believe this forms the basis of an extremely powerful modeling tool.

We demonstrated the ability to implement an XML model parser to facilitate more efficient modeling.  While the improvement the XML provided was not as great as we had hoped this does demonstrate the ability to adapt to more efficient methods of modeling.  One area of future work is to implement conversion classes between our modeling language and some of the other more popular modeling languages such as Brahms or EOFM.  This would give us additional validation options ~\cite{bolton2013litreview}.

Finally we have demonstrated the ability to extract workload metrics that are consistent with known high workload events and closely related to human workload theories such as Multiple Resource Theory, Threaded Cognition Theory, and Operator fan-out~\cite{wickens2002multiple, salvucci2008threaded, cummings2007predicting}.  We are encouraged by the predicted workload reduction seen in our second case study when we used variations in our model to represent the effects of automating a human performed task.  And we will use this idea of predicting workload to reduce actual operator workload as a seque into the future of this work.
\end{comment}

\section{Future Work}

The ultimate goal of this work is to create a UAS for WiSAR that allows a single human to perform all of the necessary tasks.  In order to achieve this goal there is still a lot of work to be done.  This thesis represents the beginning phase of this research, which is to begin comprising workload metrics.  The next phase will be to modify and validate that the metrics reflect human workload, and the last phase will be to show that the predicted workload reductions are indeed reductions in human workload.

This work focuses on two facets of human workload, cognitive resource load and decision complexity.  To predict human workload more facets must be explored.  Jared et al. \cite{moore2014modeling} are expanding on these metrics by experimenting with a temporal workload category\cite{moore2014modeling}.  They are also developing second and third order metrics within all three workload categories.  Additionally it may be necessary to expand the scope of the human workload categories before a moderately accurate measure of human workload can be obtained.

The next step will be to find the correlations between the different workload aspects in order to use the correct ratios to predict workload.  Once a metric profile for measuring human workload has been achieved, there may be many, the next phase of modifying and validating the metrics begins.



