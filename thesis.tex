\documentclass[ms,electronic,twosidetoc,letterpaper,chaptercenter,parttop,lol,lof,lot]{byumsphd}
% Author: Chris Monson
%
% This document is in the public domain
%
% Options for this class include the following (* indicates default):
%
%   phd (*) -- produce a dissertation
%   ms -- produce a thesis
%
%   electronic -- default official university option, overrides the following:
%                 - equalmargins
%
%   hardcopy -- overrides the following:
%                 - no equalmargins
%                 - twoside
%
%   letterpaper -- ignored, but helpful for the Makefile that I use
%
%   10pt -- 10 point font size
%   11pt -- 11 point font size
%   12pt (*) -- 12 point font size
%
%   lof -- produce a list of figures in the preamble (off)
%   lot -- produce a list of tables in the preamble (off)
%   lol -- produce a list of listings in the preamble (off)
%
%   layout -- show layout lines on the pages, helps with overfull boxes (off)
%   grid -- show a half-inch grid on every page, helps with printing (off)
%   separator -- print an extra instruction page between preamble and body (off)
%
%   twoside (*) -- two-sided output (margins alternate for odd and even pages,
%     blank pages inserted to ensure that chapters begin on the right side of a
%     bound copy, etc.)
%   oneside -- one-sided output (margins are the same on all pages)
%   equalmargins -- make all margins equal - ugly for binding, but compliant
%
%   twosidetoc - start two-sided margins at the TOC instead of the body.  This
%     is sometimes (oddly) required, but be aware that it will make the page
%     numbering seem screwy, e.g., the first four full sheets of paper will
%     have number i-iv (not shown, though), and the next sheets will each have
%     two numbers, one for each side.  I suspect that most people don't look at
%     the roman numerals anyway, but it is a weird requirement.
%
%   openright (*) -- force new chapters to start on an odd page
%   openany -- don't use this, it's ugly
%
%   prettyheadings -- make the section/chapter headings look nice
%   compliantheadings (*) -- make them look ugly, but compliant with standards
%
%   chaptercenter -- center the chapter headings horizontally
%   chapterleft (*) -- place chapter headings on the left
%
%   partmiddle -- Part headers are centered vertically, no other text on page
%   parttop (*) -- Part headers at top of page, other text expected
%
%   duplexprinter -- Ensures that the two-sided portion starts on the right
%     side when printing.  This is not for use in submission, since the best
%     thing to do there is to print everything out one-sided, then take it down
%     to the copy store to have them do the rest.  It does help to save trees
%     when you are printing out copies just to look at them and fiddle with
%     things.
%
%
% EXAMPLES:
%
% The rest is up to you.  To fiddle with margins, use the \settextwidth and
% \setbindingoffset macros, described below.  I suggest that you
% \settextwidth{6.0in} for better-looking output (otherwise you'll get 3/4-inch
% margins after binding, which is sort of weird).  This will depend on the
% opinions of the various dean/coordinator folks, though, so be sure to ask
% them before embarking on a major formatting task.

% The following command fixes my particular printer, which starts 0.03 inches
% too low, shifting the whole page down by that amount.  This shifts the
% document content up so that it comes out right when printed.
%
% Discovering this sort of behavior is best done by specifying the ``grid''
% option in the class parameters above.  It prints a 1/2 inch grid on every
% page.  You can then use a ruler to determine exactly what the printer is
% doing.
%
% Uncomment to shift content up (accounting for printer problems)
%\setlength{\voffset}{-.03in}

% Here we set things up for invisible hyperlinks in the document.  This makes
% the electronic version clickable without changing the way that the document
% prints.  It's useful, but optional.
%
% NOTE: "driverfallback=ps2pdf" chooses ps2pdf in the case of LaTeX and pdftex
% in the case of pdflatex. If you use my LaTeX makefile (at
% http://latex-makefile.googlecode.com/) then pdftex is the default There are
% many other benefits to using the makefile, too.  This option is not always
% available, so use with care.
\usepackage[
    bookmarks=true,
    bookmarksnumbered=true,
    breaklinks=false,
    raiselinks=true,
    pdfborder={0 0 0},
    colorlinks=false,
    plainpages=false,
    ]{hyperref}

% To fiddle with the margin settings use the below.  DO NOT change stuff
% directly (like setting \textwidth) - it will break subtle things and you'll
% be tearing your hair out.
%
% For example, if you want 1.5in equal margins, or 2in and 1in margins when
% printing, add the following below:
%
%\setbindingoffset{1.0in}
%\settextwidth{5.5in}
%
% When equalmargins is specified in the class options, the margins will be
% equal at 1.5in each: (8.5 - 5.5) / 2.  When equalmargins is not specified,
% the inner margin will be 2.0 and the outer margin will be 1.0: inner = (8.5 -
% 5.5 - 1.0) / 2 + 1.0 (the 1.0 is the binding offset).
%
% The idea is this: you determine how much space the text is going to take up,
% whether for an electronic document (equalmargins) or not.  You don't want the
% layout shifting around between printed and electronic documents.
%
% So, you specify the text width.  Then, if there is a binding offset (when
% binding your thesis, the binding takes up space - usually 0.5 inches), that
% reduces the visual space on the final printed copy.  So, the *effective*
% margins are calculated by reducing the page size by the binding offset, then
% computing the remaining space and dividing by two.  Adding back in the
% binding offset gives the inner margin.  The outer margin is just what's left.
%
% All of this is done using the geometry package, which should be manipulated
% directly at your peril.  It's best just to use the above macros to manipulate
% your margins.
%
% That said, using the geometry macro to set top and bottom margins, or
% anything else vertical, is perfectly safe and encouraged, e.g.,
%
%\geometry{top=2.0in,bottom=2.0in}
%
% Just don't fiddle with horizontal margins this way.  You have been warned.

% This makes hyperlinks point to the tops of figures, not their captions
\usepackage[all]{hypcap}

% These packages allow the bibliography to be sorted alphabetically and allow references to more than one paper to be sorted and compressed (i.e. instead of [5,2,4,6] you get [2,4-6])
\usepackage[numbers,sort&compress]{natbib}
%\usepackage{hypernat}

\usepackage{graphicx}
\usepackage{subfigure}
\usepackage{pdfpages}
\usepackage{amsmath}

% Because I use these things in more than one place, I created new commands for
% them.  I did not use \providecommand because I absolutely want LaTeX to error
% out if these already exist.
\newcommand{\Title}{Modeling Human Workload in Unmanned Aerial Systems}
\newcommand{\Author}{TJ Gledhill}
\newcommand{\GraduationMonth}{April}
\newcommand{\GraduationYear}{2014}

% Set up the internal PDF information so that it becomes part of the document
% metadata.  The pdfinfo command will display this.
\hypersetup{%
    pdftitle=\Title,%
    pdfauthor=\Author,%
    pdfsubject={MS Thesis, BYU CS Department: %
                Degree Granted \GraduationMonth~\GraduationYear, Document Created \today},%
    pdfkeywords={human machine interface, workload metrics, model checking},%
}

% Rewrite the itemize, description, and enumerate environments to have more
% reasonable spacing:
\newcommand{\ItemSep}{\itemsep 0pt}
\let\oldenum=\enumerate
\renewcommand{\enumerate}{\oldenum \ItemSep}
\let\olditem=\itemize
\renewcommand{\itemize}{\olditem \ItemSep}
\let\olddesc=\description
\renewcommand{\description}{\olddesc \ItemSep}

% Important settings for the byumsphd class.
\title{\Title}
\author{\Author}

\committeechair{Michael~A.~Goodrich}
\committeemembera{Kevin~Seppi}
\committeememberb{Eric~M.~Mercer}

\monthgraduated{\GraduationMonth}
\yeargraduated{\GraduationYear}
\yearcopyrighted{\GraduationYear}

\documentabstract{%
Unmanned aerial systems (UASs) often require multiple human operators fulfilling diverse roles for safe correct operation~\cite{GoodrichMorse2008,MurphyStoverPrattGriffin2006,Cummings2007}. Although some dispute the utility of minimizing the number of humans needed to administer a UAS~\cite{MurphyBurke2010}, minimization remains a long-standing objective for many designers.  Reliably designing the human interaction, autonomy, and decision making aspects of these systems requires the use of modeling.  We propose a conceptual model which models human machine interaction systems as a group of actors connected by a network of communication channels.  We also propose a workload taxonomy derived from a review of the relevant literature which we then apply to the conceptual model.  We present a simulation framework implemented in Java, with an optional XML model parser, which can be analyzed using the Java Pathfinder (JPF) model checker.  The simulator produces a workload profile over time for each human actor in the system.  We conducted case studies by modeling two different UAS.  Wilderness search and rescue using a UAV (WiSAR) and UAS integration into the national air space (NAS).  The results of these case studies are consistent with known workload events and the simple workload metric presented by Wickens~\cite{wickens2002multiple}.

}

\documentkeywords{%
    human workload, unmanned aerial system, uas, national air space, unmanned aerial vehicle, modeling human machine interaction
}

\acknowledgments{%
    The authors would like to thank the NSF IUCRC Center for Unmanned Aerial Systems, and the participating industries and labs, for funding the work.
    The authors would like to thank Neha Rungta of NASA Ames Intelligent
Systems Division for her help with JPF and Brahms. The authors would also like to thank the NSF IUCRC Center for Unmanned Aerial Systems, and the participating industries and labs, for funding the work. Further thanks go to Jared Moore and Robert Ivie for their help coding the model and editing this paper.
}

\department{Computer~Science}
\graduatecoordinator{Quinn~Snell}
\collegedean{Thomas~W.~Sederberg}
\collegedeantitle{Associate~Dean}

% Customize the name of the Table of Contents section.
\renewcommand\contentsname{Table of Contents}

% Remove all widows an orphans.  This is not normally recommended, but in a
% paper dissertation there is no reasonable way around it; you can't exactly
% rewrite already-published content to fix the problem.
\clubpenalty 10000
\widowpenalty 10000

% Allow pages to have extra blank space at the bottom in order to accommodate
% removal of widows and orphans.
\raggedbottom

% Produce nicely formatted paragraphs. There is nothing additional to do.  In
% case you get some problems, surround your text with
% \begin{sloppy} ... \end{sloppy}. If that does not work, try
% \microtypesetup{protrusion=false} ... \microtypesetup{protrusion=true}
\usepackage{microtype}

\begin{document}

% Produce the preamble
\microtypesetup{protrusion=false}
\maketitle
\microtypesetup{protrusion=true}


\chapter{Introduction and Overview}

Most existing Unmanned Aerial Systems (UASs) require two or more human operators~\cite{GoodrichMorse2008,MurphyStoverPrattGriffin2006}. Standard UAS practice is to have one human to control the aerial vehicle and another to control the camera or other payloads. In addition to this a third human is often responsible for overseeing task completion and interfacing with the command structure. Although some argue persuasively that this is a desirable organization~\cite{MurphyBurke2010}, there is considerable interest in reducing the required number of humans and reducing human workload using improved autonomy and enhanced user interfaces~\cite{Cummings2007,MitchellCummings2005,goodrich2010fanout}.

UAV Enabled Wilderness Search and Rescue (UE-WiSAR) has been a focus of the Human
Centered Machine Intelligence (HCMI), Multiple Agent Intelligent Coordination and Control (MAGICC) and Computer Vision (CV) labs at Brigham Young University since 2005.  In that time research has been conducted on human interaction with mUAVs, improving target detection by enhancing video taken from a mUAV, integrating mUAVs into a SAR environment,
and improving the mUAVs chance of getting video footage of the target.   Given our prior experience with WiSAR~\cite{goodrich2009towards} we proposed to construct a UAS for that domain.  The initial proposal was to move directly into software development.  See appendix~\ref{app:uas_wisar}.  During the requirement gathering and design steps of this project it became clear just how complex the system was.  While prototypes of almost all the functionality existed, there was no way of verifying that the system itself would meet the requirements.  Instead of blindly pressing forward with the software development we decided that the more important scientific problem was to figure out how to model and validate the system design before implementation.

System modeling is not a new approach, though applying system modeling to human-machine systems requires extending most modeling languages.  There are many different modeling languages each of which is designed to perform specific types of validation~\cite{bolton2013litreview}.  While it was possible to extend an existing modeling language to support our goals much like Bass et al. have done with EOFM~\cite{bass2011toward}, we chose to create our own framework for the following reasons:  
First, we wished to avoid semantically constrained languages which may not fully map to the model.  A common denominator among system modeling languages is the focus on well-defined tasks.  For new or emerging systems these detailed tasks are vague or undefined`\ref{humphrey2009information}.  We needed a highly flexible representation which allowed for varying levels of abstraction.  
Second, we desired to perform model checking directly on the modeling language itself instead of converting the language into another form for verification.
Finally, we desired to perform human workload measurements on the model as part of our approach to reducing the required number of humans involved in UASs.

The Model Abstraction Framework (MAF) consists of the following core components:  Models, Modeling Interface, Simulator, and Workload Viewer.  A link to the source code is found in appendix~\ref{code}.  MAF uses Java as the primary modeling language.  The models consist of a set of Directed Role Graphs (DiRGs) representing the entities which are performing actions and a Directed Team Graph (DiTG) representing the communication network between the different entities.  The model creates the DiRGs and DiTG by implementing the Modeling Interface, a set of Java interfaces and abstract classes, which allow the model to be expressed as a labeled state transition system (LSTS) which is run by the Simulator.  The Simulator is a core set of Java classes which simulates the running of the model and gathers workload metrics.  The simulation itself is performed inside of Java Pathfinder (JPF) which performs the model checking.  As JPF explores each path through the model it collects the workload metrics for that path.  When the simulation is complete the Workload Viewer allows us to analyze the human workload measurements, something which is relatively new to human machine interface validation~\cite{bolton2013litreview}.

We chose to base our workload measurements off of multiple resource theory~\cite{wickens2002multiple} with ties to queuing theory~\cite{newell1994unified} and operator fan-out theory~\cite{goodrich2010fanout}.  By relating these theories to the different operational components of the model we can obtain a quantitative measure of an Actors workload for each time-step in the system.

As part of this work we extended MAF by creating an XML Model Parser.  This component allowed models to be defined using XML and converted into Java.  The purpose of this exercise was two fold.  Constrain model implementations without losing  flexibility and improve the time to model.  The results of the XML Model Parser were very positive, although the time to model benefits rapidly decreased as complexity increased.

We performed two case studies, one for WiSAR and another representing the introduction of a UAS into the National Air Space (NAS).  We chose WiSAR because of the host of modeling information available to us~\cite{adams2009cognitive}.  We chose to model a UAS operating within the NAS because of the current interest in the subject~\cite{nasroadmap} and the decided lack of modeling information available to us which required us to use high levels of abstraction.  The results of the case studies show that the modeling framework we developed is capable of accurately modeling UASs.  They also demonstrate the ability to model systems using varying degrees of abstraction.  The verification of the metrics in this thesis is done by a {\em consistency} approach, showing that workload rises and falls as expected given known difficult and easy scenarios, respectively.  Future work will provide a more detailed verification.


\section{Overview and Papers}

Chapters 2 and 3 of this thesis consist of two published papers.  Chapter 2 introduces the DiRG and simulation framework.  Chapter 3 extends chapter 2 by adding the DiTG and workload metrics.

Chapter 2 presents our core conceptual model, details of MAF, and a WiSAR model implemented in Java.  It also presents how our simulator is able to validate the model using JPF.

Chapter 3 presents an extension of our conceptual model in the form of the DiTG.  It also presents a formal taxonomy of workload metrics and how that taxonomy applies to our conceptual model.  Lastly it reports the results of adding the workload metrics into the simulation framework.  Much of the work performed for this paper was completed by others, it has been added to this thesis for completeness.

Chapter 4 presents an XML Model Extension for MAF which makes it easier for the modeler to create a LSTS and reduces the likely hood of coding errors in the resulting Java implementation of the LSTS.

Chapter 5 includes the changes made to the Modeling Interface, Simulator, and workload metrics as a result of observations and experiences from the previous chapters.  

Chapter 6 presents a case study which involves modeling the integration of a UAS into the NAS.  This case study uses the new XML modeling format to produce several versions of the Java model.  Lastly a detailed analysis and comparison of the results of these models is performed.

Chapter 7 contains our conclusions and the ideas we find most appealing for future work.

The appendices include the initial WiSAR proposal, the core modeling framework classes, an xml model, and simulation logs.
\chapter{Modeling UASs for Role Fusion and Human Machine Interface Optimization}
\chapter{Modeling Human Workload in Unmanned Aerial Systems}
\chapter{Refactoring the Modeling Framework}

% \chapter{Modeling}
% 
% By far the most challenging aspect of this work was the creation of the models.  By constructing the simulation framework as a set of Java interfaces and abstract classes it meant that the framework was almost entirely extensible in any direction.  We found this very refreshing during the first few development iterations of the framework and model.  We created different Actor types and different sub-Actors within those types.  Each transition used an anonymous method containing logic for enabling the transition.  Eventually as our model grew in size and complexity the freedom of the Java language became a dual edged sword.  Our model was full of implicit declarations, inconsistencies, duplicated code, and anonymous methods.  It became extremely difficult to maintain the model let alone add to it.
% 
% \section{Making modeling easier}
% To help improve the modeling experience we chose to 
\chapter{Case Study: UAS operating in the NAS}

According to the UAS integration into the NAS roadmap~\cite{nasroadmap} the FAA is working with other government agencies and industry to develop a collaborative UAS modeling and
simulation environment to explore key challenges to UAS integration. The near-term modeling goals are to:
\begin{itemize}
  \item Validate current mitigation proposals
  \item Establish a baseline of end-to-end UAS performance measures
  \item Establish thresholds for safe and efficient introduction of UAS into the NAS
  \item And develop NextGen concepts, including 4-dimensional trajectory utilizing UAS technology
\end{itemize}

We believe that our modeling language accomplishes each of these goals.  Our modeling language is extremely flexible allowing it to model a large variety of systems.  We have the ability to detect critical failures.  We have the ability to examine multiple paths through the model and to randomly perturb the model in different ways to explore these paths.  We have the ability to model specific performance measures as extensions to the base model.  While this is left to future work the concept of a set of Actors which can be attached to a third party model to perform performance measuring is very attractive when attempting to standardize model behavior.  We also have the ability to generate human workload metrics is valuable for setting safety thresholds and analyzing new designs.  We also have the ability to use different levels of abstraction for each portion of our model.  This is ideal while attempting to develop new concepts as it allows us to predict the results of model changes without an exact understanding of how the changes will be implemented.  

The document~\cite{nasroadmap} also identifies several interrelated research challenges:
\begin{itemize}
  \item Effective human-automation interaction (level; trust; and mode awareness)
  \item Pilot-centric ground control station design (displays; sensory deficit and remediation; and sterile cockpit)
  \item Display of traffic/airspace information (separation assurance interface)
  \item Predictability and contingency management (lost link status; lost ATC communication; and ATC workload)
  \item Definition of roles and responsibilities (communication flow among crew, ATC, and flight dispatcher)
  \item System-level issues (NAS-wide human performance requirements)
  \item And airspace users� and providers� qualification and training (crew/ATC skill set, training, certification, and currency)
\end{itemize}

Our modeling language was specifically designed to target challenges 1, 4, 5, and 6.  We hope that our work may be effective in overcoming the other challenges as well but we leave this to future work.

To demonstrate the recent changes and additions to our modeling language we chose to model a basic UAS integrating with the NAS.  Due to our lack of domain knowledge we have had to make a fair number of assumptions in order to achieve a high level of abstraction.  Despite the high level of abstraction we believe that the results are still quite impressive.

\section{Model Scenario and Assumptions}

A UAS plans to operate within the NAS.  The UAV will take off and land at an airport serving both manned and unmanned aerial vehicles.  The UAV Operator is located at this airport and visually monitors the takeoff and landing of the UAV.  UAV state is continuously shown on the UAS GUI.

There is an FAA system which provides real-time notice to airmen (NOTAM) information.  The UAS connects to this system and displays these NOTAMs on a GUI.  The FAA system also allows UASs to file flight plans.  Filed flight plans are automatically checked for simple conflicts such as crossing NOTAMS or duplicate takeoff/landing times.  If there is a conflict the FAA system flags the flight plan for an ATC (Air Traffic Controller) and displays these requests on its GUI.  The ATC then approves or denies flight plans, using the FAA System GUI, at their leisure.  This approval/denial then becomes available to the UAS which displays it on its GUI.  

The FAA system also provides radar information to the ATC through its GUI.  The ATC uses this information to spot potential collisions with UAVs, it is assumed that the UAS also sends near real time position information to the FAA system although that is not required for this to work.  If a potential UAV collision is detected the ATC creates an emergency NOTAM in the region of conflict.  This emergency NOTAM is seen by the UAS.  If the UAV is in or near the emergency NOTAM it must change course to immediately evacuate/avoid the NOTAM.  This can be done automatically by the UAS or manually by the UAV Operator.  Once the UAV has finished avoiding the emergency NOTAM it enters a loiter state.  The UAV Operator is then required to change the flight plan before the UAV will leave the loiter state.  

The UAS also has a radar for the UAV which can detect nearby objects.  This information is displayed on its GUI and if the UAV Operator detects a potential collision they will begin the deconfliction procedure which requires changing the current flight plan.  Once the UAV has landed the scenario is considered complete.

\subsection{Assumptions}

The number of assumptions made in this scenario is too great to fully list.  Instead we have only attempted to list the major assumptions which are required for the model to perform as designed.
\begin{itemize}
  \item The UAV has an unlimited flight time, never loses contact with the UAS, can takeoff and land without incident, and has accurate GPS data, is non line-of-sight.
  \item The UAS never loses connection to the FAA System, all communication with the UAV and FAA System is instant, no bugs, can create flight plans, can detect NOTAMs on the flight plan, can automatically direct the UAV out of an emergency NOTAM, displays radar information from the UAV.
  \item The UAV Operator detects all warnings displayed on the UAS GUI, generates flight plans which do not touch NOTAMS, can always deconflict the UAV, never gets fatigued.
  \item The FAA System distributes NOTAMS and automatically detects if a flight plan needs to be approved by the ATC.
  \item The ATC detects all information displayed on the FAA System GUI, can add NOTAMS to the FAA System, always adds NOTAMS correctly, approves all flight plans.
\end{itemize}

\section{Building the Model}


\chapter{Conclusions and Future Work}

As part of this thesis work we have developed the Model Abstraction Framework.  This modeling framework is based on the concept of directed role graphs and directed team graphs and is capable of modeling complex human machine systems as labeled state transition systems.  We have shown that the Model Abstraction Framework is extremely flexible, allowing each portion of a system to be modeled at varying levels of abstraction.  This allows us to model system components without knowing the implementation details.  We demonstrated this in the UAV Enabled Wilderness Search and Rescue (WiSAR) case study when we found flaws in the Video Operator role which were previously undocumented and again in the Unmanned Aerial System integration into the National Airspace System case study when we demonstrated the effect of automating a portion of the UAV Operator duties.  

Additionally we have demonstrated our ability to perform model checking directly on the Model Abstraction Framework models using Java Pathfinder (JPF).  With this tool we were able to explore multiple paths through our WiSAR model to find those paths which lead to critical failure.  With the ability to perform model checking directly on our models coupled with the flexibility of our modeling language and our ability to extract workload metrics we have the basis of an extremely powerful modeling tool.

We demonstrated the ability to implement an XML model parser to facilitate more efficient modeling.  While the improvement the XML provided was not as great as we had hoped this does demonstrate the ability to adapt to more efficient methods of modeling.  One area of future work is to implement conversion classes between our modeling language and some of the other more popular modeling languages such as Brahms or EOFM which would give us additional validation options ~\cite{bolton2013litreview}.

Finally we have demonstrated the ability to extract workload metrics which are consistent with known high workload events and closely related to human workload theories such as Multiple Resource Theory, Threaded Cognition Theory, and Operator fan-out~\cite{wickens2002multiple, salvucci2008threaded, cummings2007predicting}.  We are encouraged by the predicted workload reduction seen in our second case study which segues into the future of this work.

\section{Future Work}

The ultimate goal of this work is to create a UAS for WiSAR which allows a single human to perform all of the necessary tasks.  Before this can be accomplished using the Model Abstraction Framework the workload metrics need to be verified through sensitivity and user studies, currently being pursued by other researchers.  We anticipate that machine learning algorithms will be able to take data from these to produce a set of weightings which can be applied to the individual workload metrics.  Once the workload metrics have settled the next step is to polish up the WiSAR model, add performance measurement Actors, and systematically modify the model to find a single human design which satisfies the performance measurements and maintains workload below a certain threshold.  The next step after completing the WiSAR model is to create a generic UAS model which can then be used to explore the effects of novel UAS GUI designs.

In relation to the future of the Model Abstraction Framework there are several avenues of future work which we feel would be the most beneficial.  The first is improvements to the modeling process.  We see this happening in several ways; new language structures which improve the ability to visualize the model, conversion classes which take advantage of existing language structures to generate models for our framework, or extending another modeling language such as Brahms or EOFM to work in a similar fashion.

Another avenue of future work is to establish a verification framework to facilitate the model checking process.  One way we see this working is to allow the models to specify metric thresholds, Actor-specific performance measurements, and dynamic portions of the model.  The verification framework then uses machine learning to iteratively change the dynamic portions of the model to find an optimized design.




\appendix
\input{appendix_proposal.tex}
\chapter{Java Classes} \label{java_classes}

This appendix contains the core Java classes and interfaces used by our modeling framework.

\section{Core Classes}
\begin{spacing}{.5}
	\lstinputlisting[language=Java]{Simulator.java}
	
	\lstinputlisting[language=Java]{DeltaClock.java}
	
	\lstinputlisting[language=Java]{ComChannel.java}
	
	\lstinputlisting[language=Java]{Memory.java}
	
	\lstinputlisting[language=Java]{TempComChannel.java}
	
	\lstinputlisting[language=Java]{TempMemory.java}
\end{spacing}

\section{Interfaces}
\begin{spacing}{.5}
	\lstinputlisting[language=Java]{IActor.java}
	
	\lstinputlisting[language=Java]{IComChannel.java}
	
	\lstinputlisting[language=Java]{IComLayer.java}
	
	\lstinputlisting[language=Java]{IDeltaClock.java}
	
	\lstinputlisting[language=Java]{IEvent.java}
	
	\lstinputlisting[language=Java]{IMetrics.java}
	
	\lstinputlisting[language=Java]{IState.java}
	
	\lstinputlisting[language=Java]{ITeam.java}
	
	\lstinputlisting[language=Java]{ITransition.java}
\end{spacing}

\section{Abstract Classes}
\begin{spacing}{.5}
	\lstinputlisting[language=Java]{Actor.java}
	
	\lstinputlisting[language=Java]{Event.java}
	
	\lstinputlisting[language=Java]{State.java}
	
	\lstinputlisting[language=Java]{Team.java}
\end{spacing}


\chapter{XML Model Sample} \label{XMLModel}

\begin{spacing}{.5}
	\lstinputlisting[language=XML]{UAS_in_NAS_auto_avoid_notam.xml}
\end{spacing}
\chapter{Debug Log Generated by the Model}

This is the output log generated when running the UAS integrated into the NAS model with automatic emergency NOTAM avoidance.  The main simulation cycle consists of three main parts: Load transitions, advance time, and fire transitions.


\begin{spacing}{1}
\tiny
\begin{verbatim}
Load Transition(1):	 Start a new UAV mission
Time: 1
Fired Transition:	 Start a new UAV mission
-----------------------------------------------
Load Transition(1):	UAVOP: StartState: IDLE EndState: BUILD_FP Description: Received new mission, begin building flight plan.
Time: 2
Fired Transition:	UAVOP: StartState: IDLE EndState: BUILD_FP Description: Received new mission, begin building flight plan.
-----------------------------------------------
Load Transition(900):	UAVOP: StartState: BUILD_FP EndState: BUILD_FP Description: Build a flight plan
Time: 902
Fired Transition:	UAVOP: StartState: BUILD_FP EndState: BUILD_FP Description: Build a flight plan
-----------------------------------------------
Load Transition(1):	UAVOP: StartState: BUILD_FP EndState: END_BUILD_FP Description: Click the Send flight plan button
Time: 903
Fired Transition:	UAVOP: StartState: BUILD_FP EndState: END_BUILD_FP Description: Click the Send flight plan button
-----------------------------------------------
Load Transition(1):	UAS: StartState: NORMAL EndState: NORMAL Description: Send flight plan to FAA
Load Transition(1):	UAVOP: StartState: END_BUILD_FP EndState: WAITING_ON_FAA Description: Clear mouse and keyboard output layers
Time: 904
Fired Transition:	UAS: StartState: NORMAL EndState: NORMAL Description: Send flight plan to FAA
Fired Transition:	UAVOP: StartState: END_BUILD_FP EndState: WAITING_ON_FAA Description: Clear mouse and keyboard output layers
-----------------------------------------------
Load Transition(1):	FAA: StartState: NORMAL EndState: NORMAL Description: Received Flight plan from UAS
Time: 905
Fired Transition:	FAA: StartState: NORMAL EndState: NORMAL Description: Received Flight plan from UAS
-----------------------------------------------
Load Transition(1):	ATC: StartState: NORMAL EndState: APPROVING_FLIGHT_PLAN Description: ATC is ready to approve flight plans.
Time: 906
Fired Transition:	ATC: StartState: NORMAL EndState: APPROVING_FLIGHT_PLAN Description: ATC is ready to approve flight plans.
-----------------------------------------------
Load Transition(200):	ATC: StartState: APPROVING_FLIGHT_PLAN EndState: END_APPROVING_FLIGHT_PLAN Description: ATC is approving flight plans.
Time: 1106
Fired Transition:	ATC: StartState: APPROVING_FLIGHT_PLAN EndState: END_APPROVING_FLIGHT_PLAN Description: ATC is approving flight plans.
-----------------------------------------------
Load Transition(1):	ATC: StartState: END_APPROVING_FLIGHT_PLAN EndState: NORMAL Description: ATC finished approving the flight plan.
Load Transition(1):	FAA: StartState: NORMAL EndState: NORMAL Description: FAA received approved flight plan from ATC, send to UAS
Time: 1107
Fired Transition:	ATC: StartState: END_APPROVING_FLIGHT_PLAN EndState: NORMAL Description: ATC finished approving the flight plan.
Fired Transition:	FAA: StartState: NORMAL EndState: NORMAL Description: FAA received approved flight plan from ATC, send to UAS
-----------------------------------------------
Load Transition(1):	UAS: StartState: NORMAL EndState: NORMAL Description: Received flight plan approved from FAA
Time: 1108
Fired Transition:	UAS: StartState: NORMAL EndState: NORMAL Description: Received flight plan approved from FAA
-----------------------------------------------
Load Transition(30):	UAVOP: StartState: WAITING_ON_FAA EndState: END_WAITING_ON_FAA Description: FAA approved the flight
Time: 1138
Fired Transition:	UAVOP: StartState: WAITING_ON_FAA EndState: END_WAITING_ON_FAA Description: FAA approved the flight
-----------------------------------------------
Load Transition(1):	UAS: StartState: NORMAL EndState: NORMAL Description: UAVOP sent take off command
Load Transition(1):	UAVOP: StartState: END_WAITING_ON_FAA EndState: MONITOR_TAKEOFF Description: Clear user output
Time: 1139
Fired Transition:	UAS: StartState: NORMAL EndState: NORMAL Description: UAVOP sent take off command
Fired Transition:	UAVOP: StartState: END_WAITING_ON_FAA EndState: MONITOR_TAKEOFF Description: Clear user output
-----------------------------------------------
Load Transition(1):	UAV: StartState: GROUNDED EndState: TAKEOFF Description: Move to takeoff from UAVGUI cmd
Time: 1140
Fired Transition:	UAV: StartState: GROUNDED EndState: TAKEOFF Description: Move to takeoff from UAVGUI cmd
-----------------------------------------------
Load Transition(300):	UAV: StartState: TAKEOFF EndState: FLYING Description: Automatic Transition to Flying
Load Transition(1):	UAS: StartState: NORMAL EndState: NORMAL Description: UAV has started to takeoff, show this to the UAVOp
Time: 1141
Fired Transition:	UAS: StartState: NORMAL EndState: NORMAL Description: UAV has started to takeoff, show this to the UAVOp
-----------------------------------------------
Time: 1440
Fired Transition:	UAV: StartState: TAKEOFF EndState: FLYING Description: Automatic Transition to Flying
-----------------------------------------------
Load Transition(1):	Potential collision on UAS radar
Load Transition(1):	FAA radar shows a potential collision with a UAV
Load Transition(1):	Request that ATC create a new NOTAM on UAV FP
Load Transition(10000):	UAV: StartState: FLYING EndState: LANDING Description: UAV normal flight
Load Transition(1):	UAS: StartState: NORMAL EndState: NORMAL Description: UAV has started to fly, show this to the UAVOp
Load Transition(1):	UAVOP: StartState: MONITOR_TAKEOFF EndState: MONITOR_UAVGUI Description: UAV is airborne, move to monitor GUI
Time: 1441
Fired Transition:	Request that ATC create a new NOTAM on UAV FP
Fired Transition:	Potential collision on UAS radar
Fired Transition:	UAS: StartState: NORMAL EndState: NORMAL Description: UAV has started to fly, show this to the UAVOp
Fired Transition:	FAA radar shows a potential collision with a UAV
Fired Transition:	UAVOP: StartState: MONITOR_TAKEOFF EndState: MONITOR_UAVGUI Description: UAV is airborne, move to monitor GUI
-----------------------------------------------
Load Transition(1):	ATC: StartState: NORMAL EndState: CREATE_NOTAM Description: ATC needs to create a new NOTAM
Load Transition(1):	FAA: StartState: NORMAL EndState: NORMAL Description: Potential collision on FAA Radar, show the user
Load Transition(1):	UAS: StartState: NORMAL EndState: NORMAL Description: UAS received radar collision event
Load Transition(900):	UAVOP: StartState: MONITOR_UAVGUI EndState: MONITOR_UAVGUI Description: Monitoring the UAVGUI
Time: 1442
Fired Transition:	ATC: StartState: NORMAL EndState: CREATE_NOTAM Description: ATC needs to create a new NOTAM
Fired Transition:	FAA: StartState: NORMAL EndState: NORMAL Description: Potential collision on FAA Radar, show the user
Fired Transition:	UAS: StartState: NORMAL EndState: NORMAL Description: UAS received radar collision event
-----------------------------------------------
Load Transition(1):	ATC: StartState: CREATE_NOTAM EndState: CREATE_EMERGENCY_NOTAM Description: ATC sees the alert about a UAV collision and trie...
Load Transition(1):	UAVOP: StartState: MONITOR_UAVGUI EndState: AVOID_COLLISION Description: Operator sees a potential conflict and trys to avoid...
Time: 1443
Fired Transition:	ATC: StartState: CREATE_NOTAM EndState: CREATE_EMERGENCY_NOTAM Description: ATC sees the alert about a UAV collision and tries ...
Fired Transition:	UAVOP: StartState: MONITOR_UAVGUI EndState: AVOID_COLLISION Description: Operator sees a potential conflict and trys to avoid i...
-----------------------------------------------
Load Transition(60):	ATC: StartState: CREATE_EMERGENCY_NOTAM EndState: END_CREATE_EMERGENCY_NOTAM Description: ATC is creating an Emergency Nota...
Load Transition(300):	UAVOP: StartState: AVOID_COLLISION EndState: AVOID_COLLISION Description: Operator is in the act of deconflicing the UAV
Time: 1503
Fired Transition:	ATC: StartState: CREATE_EMERGENCY_NOTAM EndState: END_CREATE_EMERGENCY_NOTAM Description: ATC is creating an Emergency Notam ar...
-----------------------------------------------
Load Transition(1):	ATC: StartState: END_CREATE_EMERGENCY_NOTAM EndState: AVOID_UAV_COLLISION Description: ATC is creating an Emergency Notam aro...
Load Transition(1):	FAA: StartState: NORMAL EndState: NORMAL Description: ATC sent an Emergency NOTAM to the FAA system, send it to the UAS
Time: 1504
Fired Transition:	ATC: StartState: END_CREATE_EMERGENCY_NOTAM EndState: AVOID_UAV_COLLISION Description: ATC is creating an Emergency Notam aroun...
Fired Transition:	FAA: StartState: NORMAL EndState: NORMAL Description: ATC sent an Emergency NOTAM to the FAA system, send it to the UAS
-----------------------------------------------
Load Transition(500):	ATC: StartState: AVOID_UAV_COLLISION EndState: AVOID_UAV_COLLISION Description: ATC is waiting for collision to stop.
Load Transition(1):	UAS: StartState: NORMAL EndState: NORMAL Description: Receive Emergency Notam on the UAV, force UAV away from NOTAM
Time: 1505
Fired Transition:	UAS: StartState: NORMAL EndState: NORMAL Description: Receive Emergency Notam on the UAV, force UAV away from NOTAM
-----------------------------------------------
Load Transition(1):	UAV: StartState: FLYING EndState: AVOID_NOTAM Description: UAV gets a command to avoid a NOTAM
Time: 1506
Fired Transition:	UAV: StartState: FLYING EndState: AVOID_NOTAM Description: UAV gets a command to avoid a NOTAM
-----------------------------------------------
Load Transition(300):	UAV: StartState: AVOID_NOTAM EndState: LOITER Description: Avoid the Emergency NOTAM
Load Transition(1):	UAS: StartState: NORMAL EndState: NORMAL Description: UAV is avoiding a NOTAM, show this to the UAVOp
Time: 1507
Fired Transition:	UAS: StartState: NORMAL EndState: NORMAL Description: UAV is avoiding a NOTAM, show this to the UAVOp
-----------------------------------------------
Time: 1743
Fired Transition:	UAVOP: StartState: AVOID_COLLISION EndState: AVOID_COLLISION Description: Operator is in the act of deconflicing the UAV
-----------------------------------------------
Load Transition(1):	UAS: StartState: NORMAL EndState: NORMAL Description: Radar collision has been avoided
Time: 1744
Fired Transition:	UAS: StartState: NORMAL EndState: NORMAL Description: Radar collision has been avoided
-----------------------------------------------
Load Transition(1):	UAVOP: StartState: AVOID_COLLISION EndState: MONITOR_UAVGUI Description: Collision has been avoided, return to watching the U...
Time: 1745
Fired Transition:	UAVOP: StartState: AVOID_COLLISION EndState: MONITOR_UAVGUI Description: Collision has been avoided, return to watching the UAV...
-----------------------------------------------
Load Transition(1):	UAVOP: StartState: MONITOR_UAVGUI EndState: AVOID_EMERGENCY_NOTAM Description: Operator notices an emergency notam on the UAV...
Time: 1746
Fired Transition:	UAVOP: StartState: MONITOR_UAVGUI EndState: AVOID_EMERGENCY_NOTAM Description: Operator notices an emergency notam on the UAV a...
-----------------------------------------------
Load Transition(300):	UAVOP: StartState: AVOID_EMERGENCY_NOTAM EndState: AVOID_EMERGENCY_NOTAM Description: Operator is waiting for the UAV to au...
Time: 1806
Fired Transition:	UAV: StartState: AVOID_NOTAM EndState: LOITER Description: Avoid the Emergency NOTAM
-----------------------------------------------
Load Transition(1):	End the FAA potential collision with a UAV
Load Transition(10000):	UAV: StartState: LOITER EndState: LANDING Description: Default to land after loitering for a long time.
Load Transition(1):	UAS: StartState: NORMAL EndState: NORMAL Description: UAV is now Loitering, show this to the UAVOp
Time: 1807
Fired Transition:	End the FAA potential collision with a UAV
Fired Transition:	UAS: StartState: NORMAL EndState: NORMAL Description: UAV is now Loitering, show this to the UAVOp
-----------------------------------------------
Load Transition(1):	FAA: StartState: NORMAL EndState: NORMAL Description: Radar collision has ended.
Load Transition(5):	UAVOP: StartState: AVOID_EMERGENCY_NOTAM EndState: AVOID_EMERGENCY_NOTAM Description: Operator sees that the UAV is loitering...
Time: 1808
Fired Transition:	FAA: StartState: NORMAL EndState: NORMAL Description: Radar collision has ended.
-----------------------------------------------
Load Transition(1):	ATC: StartState: AVOID_UAV_COLLISION EndState: NORMAL Description: FAA radar no longer shows a potential collision
Time: 1809
Fired Transition:	ATC: StartState: AVOID_UAV_COLLISION EndState: NORMAL Description: FAA radar no longer shows a potential collision
-----------------------------------------------
Load Transition(1):	ATC: StartState: NORMAL EndState: CREATE_NOTAM Description: ATC needs to create a new NOTAM
Time: 1810
Fired Transition:	ATC: StartState: NORMAL EndState: CREATE_NOTAM Description: ATC needs to create a new NOTAM
-----------------------------------------------
Load Transition(60):	ATC: StartState: CREATE_NOTAM EndState: END_CREATE_NOTAM Description: ATC is creating a new NOTAM
Time: 1812
Fired Transition:	UAVOP: StartState: AVOID_EMERGENCY_NOTAM EndState: AVOID_EMERGENCY_NOTAM Description: Operator sees that the UAV is loitering, ...
-----------------------------------------------
Load Transition(1):	UAS: StartState: NORMAL EndState: NORMAL Description: UAVOP sent resume flight
Load Transition(1):	UAVOP: StartState: AVOID_EMERGENCY_NOTAM EndState: MONITOR_UAVGUI Description: NOTAM has been avoided, return to watching the...
Time: 1813
Fired Transition:	UAS: StartState: NORMAL EndState: NORMAL Description: UAVOP sent resume flight
Fired Transition:	UAVOP: StartState: AVOID_EMERGENCY_NOTAM EndState: MONITOR_UAVGUI Description: NOTAM has been avoided, return to watching the U...
-----------------------------------------------
Load Transition(1):	UAV: StartState: LOITER EndState: FLYING Description: Resume a normal flight
Time: 1814
Fired Transition:	UAV: StartState: LOITER EndState: FLYING Description: Resume a normal flight
-----------------------------------------------
Load Transition(10000):	UAV: StartState: FLYING EndState: LANDING Description: UAV normal flight
Load Transition(1):	UAS: StartState: NORMAL EndState: NORMAL Description: UAV has started to fly, show this to the UAVOp
Time: 1815
Fired Transition:	UAS: StartState: NORMAL EndState: NORMAL Description: UAV has started to fly, show this to the UAVOp
-----------------------------------------------
Load Transition(900):	UAVOP: StartState: MONITOR_UAVGUI EndState: MONITOR_UAVGUI Description: Monitoring the UAVGUI
Time: 1870
Fired Transition:	ATC: StartState: CREATE_NOTAM EndState: END_CREATE_NOTAM Description: ATC is creating a new NOTAM
-----------------------------------------------
Load Transition(1):	ATC: StartState: END_CREATE_NOTAM EndState: NORMAL Description: ATC finished creating NOTAM return to normal
Load Transition(1):	FAA: StartState: NORMAL EndState: NORMAL Description: FAA received new NOTAM on UAV FP from ATC, send to UAS
Time: 1871
Fired Transition:	ATC: StartState: END_CREATE_NOTAM EndState: NORMAL Description: ATC finished creating NOTAM return to normal
Fired Transition:	FAA: StartState: NORMAL EndState: NORMAL Description: FAA received new NOTAM on UAV FP from ATC, send to UAS
-----------------------------------------------
Load Transition(1):	FAA: StartState: NORMAL EndState: NORMAL Description: Clear New NOTAM memory
Load Transition(1):	UAS: StartState: NORMAL EndState: NORMAL Description: Receive new NOTAM from FAA which appears on the Flight Plan
Time: 1872
Fired Transition:	FAA: StartState: NORMAL EndState: NORMAL Description: Clear New NOTAM memory
Fired Transition:	UAS: StartState: NORMAL EndState: NORMAL Description: Receive new NOTAM from FAA which appears on the Flight Plan
-----------------------------------------------
Load Transition(1):	UAVOP: StartState: MONITOR_UAVGUI EndState: AVOID_NOTAM Description: Operator notices new NOTAM(s) on the flight plan, begin ...
Time: 1873
Fired Transition:	UAVOP: StartState: MONITOR_UAVGUI EndState: AVOID_NOTAM Description: Operator notices new NOTAM(s) on the flight plan, begin ch...
-----------------------------------------------
Load Transition(600):	UAVOP: StartState: AVOID_NOTAM EndState: AVOID_NOTAM Description: Operator is changing the flight plan for a normal NOTAM
Time: 2473
Fired Transition:	UAVOP: StartState: AVOID_NOTAM EndState: AVOID_NOTAM Description: Operator is changing the flight plan for a normal NOTAM
-----------------------------------------------
Load Transition(1):	UAS: StartState: NORMAL EndState: NORMAL Description: UAVOP sent a new flight
Load Transition(1):	UAVOP: StartState: AVOID_NOTAM EndState: MONITOR_UAVGUI Description: NOTAM has been avoided, return to watching the UAVGUI
Time: 2474
Fired Transition:	UAS: StartState: NORMAL EndState: NORMAL Description: UAVOP sent a new flight
Fired Transition:	UAVOP: StartState: AVOID_NOTAM EndState: MONITOR_UAVGUI Description: NOTAM has been avoided, return to watching the UAVGUI
-----------------------------------------------
Load Transition(900):	UAVOP: StartState: MONITOR_UAVGUI EndState: MONITOR_UAVGUI Description: Monitoring the UAVGUI
Time: 3374
Fired Transition:	UAVOP: StartState: MONITOR_UAVGUI EndState: MONITOR_UAVGUI Description: Monitoring the UAVGUI
-----------------------------------------------
Time: 11814
Fired Transition:	UAV: StartState: FLYING EndState: LANDING Description: UAV normal flight
-----------------------------------------------
Load Transition(600):	UAV: StartState: LANDING EndState: GROUNDED Description: Land the UAV
Load Transition(1):	UAS: StartState: NORMAL EndState: NORMAL Description: UAV has started to land, show this to the UAVOp
Load Transition(900):	UAVOP: StartState: MONITOR_UAVGUI EndState: MONITOR_UAVGUI Description: Monitoring the UAVGUI
Time: 11815
Fired Transition:	UAS: StartState: NORMAL EndState: NORMAL Description: UAV has started to land, show this to the UAVOp
-----------------------------------------------
Load Transition(1):	UAVOP: StartState: MONITOR_UAVGUI EndState: MONITOR_LANDING Description: Operator sees that the UAV is beginning the landin...
Time: 11816
Fired Transition:	UAVOP: StartState: MONITOR_UAVGUI EndState: MONITOR_LANDING Description: Operator sees that the UAV is beginning the landing a...
-----------------------------------------------
Time: 12414
Fired Transition:	UAV: StartState: LANDING EndState: GROUNDED Description: Land the UAV
-----------------------------------------------
Load Transition(1):	UAS: StartState: NORMAL EndState: NORMAL Description: UAV is GROUNDED, show this to the UAVOp
Time: 12415
Fired Transition:	UAS: StartState: NORMAL EndState: NORMAL Description: UAV is GROUNDED, show this to the UAVOp
-----------------------------------------------
Load Transition(30):	UAVOP: StartState: MONITOR_LANDING EndState: IDLE Description: Operator sees that the UAV has landed and sets the mission...
Time: 12445
Fired Transition:	UAVOP: StartState: MONITOR_LANDING EndState: IDLE Description: Operator sees that the UAV has landed and sets the mission to ...
-----------------------------------------------
Time: 12445
-----------------------------------------------
\end{verbatim}
\end{spacing}


\bibliographystyle{plainnat}
\bibliography{references}

\end{document}

% vim: lbr
